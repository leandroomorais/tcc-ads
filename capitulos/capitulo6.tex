\chapter{CONCLUSÃO}

Este trabalho foi desenvolvido com o propósito de produzir um Sistema de Gestão Hospitalar que facilitasse a gestão destes estabelecimentos com base nas suas especificidades e necessidades. O embasamento teórico buscado, aliado às metodologias e ferramentas utilizadas propiciaram a contemplação dos objetivos definidos e o desenvolvimento de um \textit{software} para plataforma web que faz uso de conceitos de responsividade, usabilidade, segurança, integridade das informações e integração com outros sistemas através dos Serviços Web.

Os resultados apresentados mostraram que o sistema produzido é capaz de se comunicar com o Barramento do CNS, o que facilita a coleta de dados do cidadão, também organiza o fluxo de atendimento e de prioridades na fila de atendimento de um hospital público através da classificação de risco, registra os dados dos atendimentos e procedimentos realizados e gera arquivo compatível com o SIA/SUS que elimina a necessidade de digitação dos Boletins de Produção Ambulatorial no BPA-MAGNÉTICO.

A cada etapa do desenvolvimento, partes do sistema eram apresentadas e testadas por profissionais de saúde do setor hospitalar e tiveram boa aceitação. A ideia inicial era criar parcerias com hospitais para ampliar os testes e buscar melhorias no software, porém devido a Pandemia causada pelo coronavírus, que deixou em alerta a rede pública de saúde e preconizou o distanciamento social, tais ações não foram possíveis.

Também durante o desenvolvimento, foram encontradas dificuldades na integração com os sistemas de informações do SUS e na compreensão dos parâmetros enviados e recebidos por estes, sendo estas contornadas a partir da leitura de diversos manuais e documentações e do auxílio de profissionais da área.

Vários ajustes e melhorias poderão ser implementados em projetos futuros ou revisões deste trabalho, como por exemplo a adição do módulo de internações hospitalares capaz de se integrar ao SIH/SUS (Sistema de Informações Hospitalares do SUS) e automatizar a geração das guias de Autorização de Internação Hospitalar - AIH.


Diante disso, espera-se que este trabalho seja relevante para área de gestão hospitalar e contribua no ganho de eficiência e rapidez dos serviços prestados pelos Hospitais Públicos, possibilitando melhorias no desempenho da gestão, auxiliando no processo de tomadas de decisões e no desenvolvimento e melhoria da qualidade dos atendimentos realizados.



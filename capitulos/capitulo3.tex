\chapter{Metodologia}

Neste capítulo serão discutidos os métodos que permitiram conhecer as necessidades e as regras de negócio envolvidas no desenvolvimento do Sistema de Gestão
Hospitalar.

Uma das primeiras ações antes de planejar o desenvolvimento do sistema proposto, foi a busca de profissionais da saúde, com experiência em gestão hospitalar, para entender o funcionamento dos hospitais, bem como verificar se realmente era viável a implementação do SGH.
Um dos passos tomados nesta etapa foi a realização de uma reunião com um auditor e consultor Técnico em Gestão dos Serviços de Saúde e alunos do curso de enfermagem da Universidade do Estado do Rio Grande do Norte, UERN. Foram realizadas ainda, visitas in loco ao
Hospital da cidade de Major Sales/RN e mantidos diálogos com gestores experientes na área de saúde,
diretores de hospitais públicos e operadores dos SIS, onde foi possível compreender melhor as regras de negócio para o funcionamento destes estabelecimentos e as principais
dificuldades enfrentadas pelos seus administradores.

A partir dos encontros realizados, foi possível confirmar a necessidade que se tinha visto anteriormente, sendo identificados os problemas no repasse de dados referentes ao Boletim de Produção Ambulatorial, principalmente no que diz respeito a digitação no aplicativo BPA-MAGNÉTICO, ocorrendo relatos de perda de produção devido à alta demanda e os prazos para alimentação do SIA. Um outro ponto observado foi o fato de o Prontuário do Paciente ser armazenado em guias e pastas de papel manuscritas, o que torna difícil a localização e a compreensão das informações registradas. Através das informações colhidas foi realizado um levantamento dos requisitos do sistema e estudou-se a viabilidade de uma ferramenta que pudesse de forma prática, segura e dinâmica atender essas necessidades e resolver parte desses problemas.

Posteriormente foi realizada uma revisão bibliográfica, onde foi conferido na literatura projetos com características semelhantes, estudo de arquiteturas de software, conceitos e tecnologias aplicadas ao sistema e o entendimento sobre a integração com os sistemas do SUS para informar a produtividade dos hospitais.

Por conseguinte, a fim de atingir o objetivo geral definido neste trabalho, foi utilizado o método de desenvolvimento ágil \textit{Scrum} para realizar a implementação do sistema, sendo realizado nesta etapa a codificação do \textit{software} com entregas incrementais avaliadas por profissional de gestão em saúde, o que permitia a alteração das partes que fugiam dos requisitos e funcionalidades especificadas.


